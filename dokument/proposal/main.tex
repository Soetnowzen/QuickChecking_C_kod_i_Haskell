\documentclass[10]{article}
%\documentclass[10,twocolumn]{article}
\usepackage[utf8]{inputenc}
\usepackage{url}
%\usepackage{hyperref}
%\usepackage{hyperref}

% \title{QuickCHecking C kod i Haskell}
\title{QuickChecking C code in Haskell}
\author{Sebastian Lagerman \\ seblag@student.chalmers.se}
% \author{Sebastian Lagerman \\ \href{mailto:seblag@student.chalmers.se}{seblag@student.chalmers.se}}
\date{October 2016}

\usepackage{natbib}
\usepackage{graphicx}

\begin{document}

\maketitle

% Backgrounds:

% Intro. FP, Datastructures, C, AFP

\section{Introduction}
% Briefly describe and motivate the project, and convince the reader of the importance of the proposed thesis work.
% A good introduction will answer these questions:
% Why is addressing these challenges significant for gaining new knowledge in the studied domain?
% How and where can this new knowledge be applied?

Testing of your code is important, but people don't want to spend an equal amount of time to write test for the same amount of code.
One way to solve this problem is to generate tests.

The goal of this proposal is to build a tool or application that one can use to QuickCheck C code against a Haskell specification.
Then use this to test open-source libraries in C, for example for data structures, image manipulation, compression, encryption, etc.
This tool will be a good way to streamline the testing to ensure software quality.
Knowledge of this will hopefully be useful in code testing to reduce the amount of time spent on writing tests for code.
For this Master thesis it will be tested on open source C code.
% \ref{QuickCheckAUTOSAR}



\section{Context}

% Use one or two high quality references for providing evidence from the literature that the proposed study indeed includes scientific and engineering challenges, or is related to existing ones.
% Convince the reader that the problem addressed in this thesis has not been solved prior to this project.

Prior work has already been done towards QuickChecking libraries in Erlang \citep{QuickCheckAUTOSAR}.
However, this particular tool utilises Erlang instead of Haskell and isn't an open-source project.
% non commerce use?

Similarly there exists property based testing libraries in C \citep{theft}.
This leaves us with a very interesting options, to extend Haskells QuickChecking to be able to test C libraries.


\section{Goals and Challenges}

% Describe your contribution with respect to concepts, theory and technical goals.
% Ensure that the scientific and engineering challenges stand out so that the reader can easily recognise that you are planning to solve an advanced problem.

The goal of the project is to create a tool for generating boiler plate code for tests for C libraries.
The tool should be easy to use for a developer with knowledge in both C and Haskell development.

% Another way to solve this problem would be to create and use a property-based library in C, however this has already been done by Scott Vokes \citep{theft}.
% To avoid reinventing the wheel and to improve Chalmers own QuickCheck we propose to use Haskell instead.
% This would make the goal to provide a Haskell module for QuickChecking properties for the following C libraries:
The module can be tested on the following open source libraries to begin with to later move on to harder libraries with structs.

\begin{itemize}
  % \item limits
  \item cmath
  \item string
  \item time
  % \item uchar
  % \item wchar
  % \item wctype
\end{itemize}

These were chosen because they are commonly used libraries in C development \citep{website:en.wikipedia.org} which also only uses primitive types in C.
% It will be good to use some standard libraries to start with to make sure that the problems that are found in the beginning actually occur inside the testing module that will be created.
More libraries may be added if time permits. % such as a library that contains bugs.

Some possible libraries that could be created with structs in them for testing the generation of types, could be:

\begin{itemize}
  % \item data structures
  % \item image manipulation
  \item compression
  \item encryption
\end{itemize}

The module will be tested and verified with QuickCheck though the properties will be found during development.


\section{Approach}

% Various scientific approaches are appropriate for different challenges and project goals.
% Outline and justify the ones that you have selected.
% For example, when your project considers systematic data collection, you need to explain how you will analyze the data, in order to address your challenges and project goals.

% One scientific approach is to use formal models and rigorous mathematical argumentation to address aspects like correctness and efficiency.
% If this is relevant, describe the related algorithmic subjects, and how you plan to address the studied problem.
% For example, if your plan is to study the problem from a computability aspect, address the relevant issues, such as algorithm and data structure design, complexity analysis, etc.
% If you plan to develop and evaluate the prototype, briefly describe your plans to design, implement, and evaluate your prototype by reviewing at most two relevant issues, such as key functionalities and their evaluation criteria.{Approach}

% \begin{itemize}
%   \item The design and implementation should specify prototype properties, such as functionalities and performance goals, e.g., scalability, memory, energy.
%     Motivate key design, with respect to state of the art and existing platforms, libraries, etc.
%   \item When discussing evaluation criteria, describe the testing environment, e.g., test-bed experiments, simulation, and user studies, which you plan to use when assessing your prototype.
%     Specify key tools, and preliminary test-case scenarios.
%     Explain how and why you plan to use the evaluation criteria in order to demonstrate the functionalities and design goals.
%     Explain how you plan to compare your prototype to the state of the art using the proposed test-case evaluation scenarios and benchmarks.
% \end{itemize}

We will create a structure for generating boiler plate code from C libraris to try and streamline the process of testing libraries.
The properties will the user of this module have to write.
Once at least one property have been written then the library will be able to be tested.
Though if the library contains a struct more work from the user will be needed.
That work will be to construct the arbitrary instance of these structs.
% We will implement the tool which will analyse a library and use QuickCheck to test the properties.

Before starting the implementation of the tool, we will need to dig deeper into the inner workings of QuickCheck and the C parsing module inline-c.
An advantage of using inline-c would be that the instance Integral is already defined for the standard types in C.
Which will make writing the Arbitrary instance for these will be simple.
Since QuickCheck is written in Haskell then the properties will be written in the same language.
% However a possibility would be to make the properties written in C as well for ease of usage.
An example of how the generated file could look like based on the library math with only the abs function in it, would be:

\begin{verbatim}
  import Foreign.C.Types
  import Test.QuickCheck
  import qualified Language.C.Inline as C
    
  C.include "<math.h>"

  instance Arbitrary CInt where
    arbitrary = arbitrarySizedIntegral
    shrink    = shrinkIntegral

  c_abs :: CInt -> CInt
  c_abs i = [C.pure| int{ abs( $(int i) ) } |] :: CInt
    
  -- Write your properties here
  -- begin each property function with "prop_"
  prop_negateValue :: CInt -> Bool
  prop_negateValue a = c_abs a == c_abs (-a)

  return []
  runTests = $quickCheckAll

  main :: IO ()
  main = runTests >>= print
\end{verbatim}

% In order to show the correctness of the tool's output, we will construct a test suite containing unit tests for each library listed above.
In order to show the correctness of the tool, we will be using QuickCheck.
The output of the generated files will be using the same output printout as QuickCheck.
For example:

Successful:

\begin{verbatim}
  Main> quickCheckingC prop_negateValue
  OK: passed 100 tests.
\end{verbatim}

Bug found:

\begin{verbatim}
  Main> quickCheckingC prop_negateValue
  Falsifiable, after 1 tests:
  [2]
  [-2]
\end{verbatim}



\section{Time Plan}

% Use one or two high quality references for providing evidence from the literature that the proposed study indeed includes scientific and engineering challenges, or is related to existing ones.
% Convince the reader that the problem addressed in this thesis has not been solved prior to this project.

\subsection{Week 13-15}

These weeks will be spent researching QuickCheck and inline-c.
We will read relevant papers (such as \citep{QuickCheck} and \citep{QuickCheckQuviq}) and analyze them to create a module for stateful testing in Haskells QuickCheck.
This is mainly preparations to learn the tools we need to use.

\subsection{Week 15-18}

With the information we gathered from the previous weeks then we will create a stateful testing module.

Preparations for the first seminar will also be made (watching the required videos and completing the exercises).

\subsection{Week 18-28}

After which we will create another module for generating boiler plate code for stateful testing of C code.

\subsection{Week 28-31}

During these weeks, we'll work on finishing the project report. The main content has been added previous weeks, so this will mainly include formatting, reference additions and other minor things.

\subsection{Week 23}

Submit halftime report.

\subsection{Week 32}

Submit opponent.

\subsection{Week 33}

Present $\&$ opposition.

\subsection{Week 35}

Submit final thesis.

\subsection{Noteworthy dates:}

\begin{description}
  \item [13 Mars 2017] - Start of project
  \item [19 April 2017] - Seminar 1
  \item [5 May 2017] - Seminar 2, Presentation
\end{description}

% Prior work has already been done towards QuickChecking other languages in Erlang \citep{QuickCheckAUTOSAR}.
% One of QuviQ services is to property based test code through their Erlang QuickCheck.
% However, this particular tool utilises Erlang instead of Haskell and isn't an open-source project. % non commerce use?

% Similarly there exists property based testing libraries in C \citep{theft}.
% This leaves us with a very interesting options, to extend Haskells module QuickCheck to be able to test C libraries.
% More importantly to generate the boiler plate code.


% \section{References}

Reference all sources that are cited in your proposal using, e.g.\ the APA, Harvard$^{2}$, or IEEE$^{3}$ style

``Testing this reference'' \citep{website:blog.helium.com}


% \section{Introduction}
% There is a theory which states that if ever anyone discovers exactly what the Universe is for and why it is here, it will instantly disappear and be replaced by something even more bizarre and inexplicable.
% There is another theory which states that this has already happened.
% \section{Introduction}
% Briefly describe and motivate the project, and convince the reader of the importance of the proposed thesis work.
% A good introduction will answer these questions:
% Why is addressing these challenges significant for gaining new knowledge in the studied domain?
% How and where can this new knowledge be applied?

Testing of your code is important, but people don't want to spend an equal amount of time to write test for the same amount of code.
One way to solve this problem is to generate tests.

The goal of this proposal is to build a tool or application that one can use to QuickCheck C code against a Haskell specification.
Then use this to test open-source libraries in C, for example for data structures, image manipulation, compression, encryption, etc.
This tool will be a good way to streamline the testing to ensure software quality.
Knowledge of this will hopefully be useful in code testing to reduce the amount of time spent on writing tests for code.
For this Master thesis it will be tested on open source C code.
% \ref{QuickCheckAUTOSAR}



% \begin{figure}[h!]
% \centering
% \includegraphics[scale=1.7]{universe.jpg}
% \caption{The Universe}
% \label{fig:univerise}
% \end{figure}

% \nocite{*}

% ``test this citation'' \citep{QuickCheckAUTOSAR}

\bibliographystyle{plain}
\bibliography{references}
\end{document}

