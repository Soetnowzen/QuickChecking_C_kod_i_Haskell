\section{Introduction}
% Briefly describe and motivate the project, and convince the reader of the importance of the proposed thesis work.
% A good introduction will answer these questions:
% Why is addressing these challenges significant for gaining new knowledge in the studied domain?
% How and where can this new knowledge be applied?

Testing of your code is important, but people don't want to spend an equal amount of time to write test for the same amount of code.
One way to solve this problem is to generate tests.

The goal of this proposal is to build a tool or application that one can use to QuickCheck C code against a Haskell specification.
Then use this to test open-source libraries in C, for example for data structures, image manipulation, compression, encryption, etc.
This tool will be a good way to streamline the testing to ensure software quality.
Knowledge of this will hopefully be useful in code testing to reduce the amount of time spent on writing tests for code.
For this Master thesis it will be tested on open source C code.
% \ref{QuickCheckAUTOSAR}

