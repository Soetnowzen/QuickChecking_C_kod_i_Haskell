\section{Introduction}
% Briefly describe and motivate the project, and convince the reader of the importance of the proposed thesis work.
% A good introduction will answer these questions:
% Why is addressing these challenges significant for gaining new knowledge in the studied domain?
% How and where can this new knowledge be applied?

Testing of code is important and a way of testing code is to generate tests and testing properties.
A useful property based testing tool is QuickCheck however currently it cannot check C libraries without other modules. % Maybe reference?
An interesting problem could then be to streamline the process of testing C code in the QuickCheck module in Haskell.
% An interesting extension to QuickCheck could then be to parse a C library and generate the boiler plate code for testing the library with the QuickCheck module in Haskell.



% build a tool to parse C files to construct the boiler plate code for the module inline-c.

% The goal of this proposal is to build a tool or application that one can use to QuickCheck C code against a Haskell specification.
% Then use this to test open-source libraries in C, for example for data structures, image manipulation, compression, encryption, etc.
% This tool will be a good way to streamline the testing to ensure software quality.
% Knowledge of this will hopefully be useful in code testing to reduce the amount of time spent on writing tests for code.
% For this Master thesis it will be tested on open source C code.
% \ref{QuickCheckAUTOSAR}

