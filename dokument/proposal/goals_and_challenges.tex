\section{Goals and Challenges}

% Describe your contribution with respect to concepts, theory and technical goals.
% Ensure that the scientific and engineering challenges stand out so that the reader can easily recognise that you are planning to solve an advanced problem.

The goal of the project is to create a tool for generating boiler plate code for tests for C libraries.
The tool should be easy to use for a developer with knowledge in both C and Haskell development.

% Another way to solve this problem would be to create and use a property-based library in C, however this has already been done by Scott Vokes \citep{theft}.
% To avoid reinventing the wheel and to improve Chalmers own QuickCheck we propose to use Haskell instead.
% This would make the goal to provide a Haskell module for QuickChecking properties for the following C libraries:
The module can be tested on the following open source libraries to begin with to later move on to harder libraries with structs.

\begin{itemize}
  % \item limits
  \item cmath
  \item string
  \item time
  % \item uchar
  % \item wchar
  % \item wctype
\end{itemize}

These were chosen because they are commonly used libraries in C development \citep{website:en.wikipedia.org} which also only uses primitive types in C.
% It will be good to use some standard libraries to start with to make sure that the problems that are found in the beginning actually occur inside the testing module that will be created.
More libraries may be added if time permits. % such as a library that contains bugs.

Some possible libraries that could be created with structs in them for testing the generation of types, could be:

\begin{itemize}
  % \item data structures
  % \item image manipulation
  \item compression
  \item encryption
\end{itemize}

The module will be tested and verified with QuickCheck though the properties will be found during development.
