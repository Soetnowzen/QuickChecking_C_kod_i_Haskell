\section{Approach}

% Various scientific approaches are appropriate for different challenges and project goals.
% Outline and justify the ones that you have selected.
% For example, when your project considers systematic data collection, you need to explain how you will analyze the data, in order to address your challenges and project goals.

% One scientific approach is to use formal models and rigorous mathematical argumentation to address aspects like correctness and efficiency.
% If this is relevant, describe the related algorithmic subjects, and how you plan to address the studied problem.
% For example, if your plan is to study the problem from a computability aspect, address the relevant issues, such as algorithm and data structure design, complexity analysis, etc.
% If you plan to develop and evaluate the prototype, briefly describe your plans to design, implement, and evaluate your prototype by reviewing at most two relevant issues, such as key functionalities and their evaluation criteria.{Approach}

% \begin{itemize}
%   \item The design and implementation should specify prototype properties, such as functionalities and performance goals, e.g., scalability, memory, energy.
%     Motivate key design, with respect to state of the art and existing platforms, libraries, etc.
%   \item When discussing evaluation criteria, describe the testing environment, e.g., test-bed experiments, simulation, and user studies, which you plan to use when assessing your prototype.
%     Specify key tools, and preliminary test-case scenarios.
%     Explain how and why you plan to use the evaluation criteria in order to demonstrate the functionalities and design goals.
%     Explain how you plan to compare your prototype to the state of the art using the proposed test-case evaluation scenarios and benchmarks.
% \end{itemize}

We will create a structure for generating boiler plate code from C libraris to try and streamline the process of testing libraries.
The properties will the user of this module have to write.
Once at least one property have been written then the library will be able to be tested.
Though if the library contains a struct more work from the user will be needed.
That work will be to construct the arbitrary instance of these structs.
% We will implement the tool which will analyse a library and use QuickCheck to test the properties.

Before starting the implementation of the tool, we will need to dig deeper into the inner workings of QuickCheck and the C parsing module inline-c.
An advantage of using inline-c would be that the instance Integral is already defined for the standard types in C.
Which will make writing the Arbitrary instance for these will be simple.
Since QuickCheck is written in Haskell then the properties will be written in the same language.
% However a possibility would be to make the properties written in C as well for ease of usage.
An example of how the generated file could look like based on the library math with only the abs function in it, would be:

\begin{verbatim}
  import Foreign.C.Types
  import Test.QuickCheck
  import qualified Language.C.Inline as C
    
  C.include "<math.h>"

  instance Arbitrary CInt where
    arbitrary = arbitrarySizedIntegral
    shrink    = shrinkIntegral

  c_abs :: CInt -> CInt
  c_abs i = [C.pure| int{ abs( $(int i) ) } |] :: CInt
    
  -- Write your properties here
  -- begin each property function with "prop_"
  prop_negateValue :: CInt -> Bool
  prop_negateValue a = c_abs a == c_abs (-a)

  return []
  runTests = $quickCheckAll

  main :: IO ()
  main = runTests >>= print
\end{verbatim}

% In order to show the correctness of the tool's output, we will construct a test suite containing unit tests for each library listed above.
In order to show the correctness of the tool, we will be using QuickCheck.
The output of the generated files will be using the same output printout as QuickCheck.
For example:

Successful:

\begin{verbatim}
  Main> quickCheckingC prop_negateValue
  OK: passed 100 tests.
\end{verbatim}

Bug found:

\begin{verbatim}
  Main> quickCheckingC prop_negateValue
  Falsifiable, after 1 tests:
  [2]
  [-2]
\end{verbatim}

