\section{Time Plan}

% Use one or two high quality references for providing evidence from the literature that the proposed study indeed includes scientific and engineering challenges, or is related to existing ones.
% Convince the reader that the problem addressed in this thesis has not been solved prior to this project.

\subsection{Week 13-15}

These weeks will be spent researching QuickCheck and inline-c.
We will read relevant papers (such as \citep{QuickCheck} and \citep{QuickCheckQuviq}) and analyze them to create a module for stateful testing in Haskells QuickCheck.
This is mainly preparations to learn the tools we need to use.

\subsection{Week 15-18}

With the information we gathered from the previous weeks then we will create a stateful testing module.

Preparations for the first seminar will also be made (watching the required videos and completing the exercises).

\subsection{Week 18-28}

After which we will create another module for generating boiler plate code for stateful testing of C code.

\subsection{Week 28-31}

During these weeks, we'll work on finishing the project report. The main content has been added previous weeks, so this will mainly include formatting, reference additions and other minor things.

\subsection{Week 23}

Submit halftime report.

\subsection{Week 32}

Submit opponent.

\subsection{Week 33}

Present $\&$ opposition.

\subsection{Week 35}

Submit final thesis.

\subsection{Noteworthy dates:}

\begin{description}
  \item [13 Mars 2017] - Start of project
  \item [19 April 2017] - Seminar 1
  \item [5 May 2017] - Seminar 2, Presentation
\end{description}

% Prior work has already been done towards QuickChecking other languages in Erlang \citep{QuickCheckAUTOSAR}.
% One of QuviQ services is to property based test code through their Erlang QuickCheck.
% However, this particular tool utilises Erlang instead of Haskell and isn't an open-source project. % non commerce use?

% Similarly there exists property based testing libraries in C \citep{theft}.
% This leaves us with a very interesting options, to extend Haskells module QuickCheck to be able to test C libraries.
% More importantly to generate the boiler plate code.
